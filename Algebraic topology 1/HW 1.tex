\documentclass[]{book}

%These tell TeX which packages to use.
\usepackage{array,epsfig}
\usepackage{amsmath}
\usepackage{amsfonts}
\usepackage{amssymb}
\usepackage{amsxtra}
\usepackage{amsthm}
\usepackage{mathrsfs}
\usepackage{color}
\usepackage{pifont}
\usepackage{mathtools}

\DeclarePairedDelimiter{\paren}{\lparen}{\rparen}

%Here I define some theorem styles and shortcut commands for symbols I use often
\theoremstyle{definition}
\newtheorem{defn}{Definition}
\newtheorem{thm}{Theorem}
\newtheorem{cor}{Corollary}
\newtheorem*{rmk}{Remark}
\newtheorem{lem}{Lemma}
\newtheorem*{joke}{Joke}
\newtheorem{ex}{Example}
\newtheorem*{soln}{Solution}
\newtheorem{prop}{Proposition}

\newcommand{\lra}{\longrightarrow}
\newcommand{\ra}{\rightarrow}
\newcommand{\surj}{\twoheadrightarrow}
\newcommand{\graph}{\mathrm{graph}}
\newcommand{\bb}[1]{\mathbb{#1}}
\newcommand{\Z}{\bb{Z}}
\newcommand{\Q}{\bb{Q}}
\newcommand{\R}{\bb{R}}
\newcommand{\C}{\bb{C}}
\newcommand{\N}{\bb{N}}
\newcommand{\M}{\mathbf{M}}
\newcommand{\m}{\mathbf{m}}
\newcommand{\MM}{\mathscr{M}}
\newcommand{\HH}{\mathscr{H}}
\newcommand{\Om}{\Omega}
\newcommand{\Ho}{\in\HH(\Om)}
\newcommand{\bd}{\partial}
\newcommand{\del}{\partial}
\newcommand{\bardel}{\overline\partial}
\newcommand{\textdf}[1]{\textbf{\textsf{#1}}\index{#1}}
\newcommand{\img}{\mathrm{img}}
\newcommand{\ip}[2]{\left\langle{#1},{#2}\right\rangle}
\newcommand{\inter}[1]{\mathrm{int}{#1}}
\newcommand{\exter}[1]{\mathrm{ext}{#1}}
\newcommand{\cl}[1]{\mathrm{cl}{#1}}
\newcommand{\ds}{\displaystyle}
\newcommand{\vol}{\mathrm{vol}}
\newcommand{\cnt}{\mathrm{ct}}
\newcommand{\osc}{\mathrm{osc}}
\newcommand{\LL}{\mathbf{L}}
\newcommand{\UU}{\mathbf{U}}
\newcommand{\support}{\mathrm{support}}
\newcommand{\AND}{\;\wedge\;}
\newcommand{\OR}{\;\vee\;}
\newcommand{\Oset}{\varnothing}
\newcommand{\st}{\ni}
\newcommand{\wh}{\widehat}

%Pagination stuff.
\setlength{\topmargin}{-.3 in}
\setlength{\oddsidemargin}{0in}
\setlength{\evensidemargin}{0in}
\setlength{\textheight}{9.in}
\setlength{\textwidth}{6.5in}
\pagestyle{empty}



\begin{document}


\begin{center}
{\Large Math 751 \hspace{0.5cm} HW 1}\\
\textbf{Jiaxi Huang}\\ %You should put your name here
Due: September 18 %You should write the date here.
\end{center}

\vspace{0.2 cm}






Chapter 0.1:
\begin{proof}
    We define $\mathbb{T}^2$ as $[-1,1]\times[-1,1]/(-1,y)=(1,y),(x,-1)=(x,1)$. And, we denote this quotient map as $p$. Here we may as well just assume the point which is removed from $\mathbb{T}^2$ is $p((0,0))$. We first define a deformation retract $F(x,t)$ on $([-1,1]\times[-1,1] \backslash (0,0))\times[0,1] $ to $Bd([-1,1]\times[-1,1])$. Let $s(x,y)= \max(\vert x\vert , \vert y\vert )$. Let $F\big((x,y),t\big)= \bigg( (1-t)x+t\frac{x}{s(x,y)}, (1-t)y+t\frac{y}{s(x,y)}  \bigg)  $. Since $(x,y) \neq 0$, this is a continuous map. Also, we have $F\big(  (x,y),0  \big)= id, F\big(  (x,y),1  \big)= Bd([-1,1]\times[-1,1]), F\big(  (x,y),t  \big)\vert_{Bd([0,1]\times[0,1])}=id $. So, this is a deformation retract. Then  for every element $e \in \mathbb{T}^2$, we let $\tilde{F}= p \circ F\bigg(  \big(  p^{-1}(e), t   \big)  \bigg): \mathbb{T}^2\times[0.1] \to \mathbb{T}^2$. This is a well-defined and continuous map, which satisfies all the requirements of being a deformation retract. So, $\tilde{F}$ is a deformation retract from $\mathbb{T}^2$ to its longitude and meridian circles.
\end{proof}

Chapter 0.2:
\begin{proof}
Let $x= (x_1, x_2, \cdots, x_n)\in \mathbb{R}^n\backslash(0,\cdots,0)$. Let $F(x,t): \mathbb{R}^n\backslash(0,\cdots,0) \to \mathbb{S}^{n-1} $ be $F(x,t)= (1-t)x+t\frac{x}{\| x\|}, \|x\|$ is $x$'s euclidean norm. Then $F$ is continuous and $F(x,0)=id$. Also, $F(x,1)=\mathbb{S}^{n-1}, F(x,t)\vert_{\mathbb{S}^{n-1}}= id $. So, this is a deformation retract from $\mathbb{R}^n\backslash(0,\cdots,0)$ to $\mathbb{S}^{n-1}$.
\end{proof}

Chapter 0.3: \\
(a)
\begin{proof}
    Let $f: X\to Y$, $g: Y \to X$, $g \circ f \simeq id|_X$, and the according homotopy is $F_1 : X\times I \to X$. Also, $f \circ g \simeq id|_Y$, and the according homotopy is $F_2 : Y\times I \to Y$. Let $f^{'}: Y\to Z$, $g^{'}: Z \to Y$, $g^{'} \circ f^{'} \simeq id|_Y$, and the according homotopy is $F_1^{'} : Y\times I \to Y$. Also, $f^{'} \circ g^{'} \simeq id|_Z$, and the according homotopy is $F_2^{'} : Z\times I \to Z$. Now we consider $f^{'} \circ f: X \to Z$ and $g\circ g^{'}:Z\to X$. First we prove a lemma. If $g: X \to Y, \ g^{'}: X\to Y,\ \ f: Y\to Z$ are continuous map. And, $g\simeq g^{'}$. We claim $f\circ g\simeq f \circ g^{'}$. Let $F: X\times I \to Y$ be the homotopy between $g$ and $g^{'}$. Then, we let $G: X\times I \to Z= f \circ F$. So, this is a continuous map. And, $G|_{X\times \{0\}}= f \circ F(X,0)=f \circ g$. Also, $G|_{X\times \{1\}}= f \circ F(X,1)=f \circ g^{'}$. So, $G$ is a homotopy between $f\circ g$ and $f\circ g^{'}$. We prove a second lemma. Let $f,g,h: X\to Y$ are continuous map, and $f\simeq g, g\simeq h$. We claim $f\simeq h$. Let $F: X\times I \to Y$ be the homotopy between $f$ and $g$, and $G: X\times I \to Y$ be the homotopy between $g$ and $h$. Let $ H(x,t)= \begin{cases}
         F(x,2t)\ \ \ 0\leq t\leq \frac{1}{2}\\
         G(x,2t-1) \ \ \frac{1}2{\leq t\leq 1}
    \end{cases} $. This is a continuous map. It is because $X\times[0,\frac{1}2{}], X\times [\frac{1}{2}, 1]$ are two closed subset of $X$, whose union is exactly $X$. And, we have $H|_{X\times [0, \frac{1}{2}]}, H|_{X\times [\frac{1}{2}, 1]}$ are continuous. Also we have $H(X,0)=f, H(X,1)=h$. so, this is a homotopy between $f$ and $h$.  Now we go back to the main proof. By our lemmas, $g \circ g^{'} \circ f^{'} \circ f=g \circ (g^{'} \circ f^{'}) \circ f \simeq g\circ id_Y\circ f= g\circ f \simeq id_X$. On the other hand, $f^{'}\circ f \circ g \circ g^{'}=f^{'}\circ (f \circ g) \circ g^{'} \simeq f^{'} \circ id_Y \circ g^{'}=f^{'}\circ g^{'} \simeq id_Z$. So, homotopy equivalence is an equivalence relation.   
\end{proof}
\\
(b)\begin{proof}
    (i) We prove reflexivity: Let $f: X\to Y$. We define $F: X\times I \to Y$ as $F(x,t)=f(x)$ which is continuous. Also, $F(X,0)=F(X,1)=f(x)$.\\
    (ii) Symmetry: Let $f,g : X\to Y$ be continuous map, and $F: X\times I \to Y$ be the homotopy between $f$ and $g$. We define $G(x,t): X\times I \to Y = F(x, 1-t)$, which is a homotopy between $g$ and $f$.\\  (iii) Transitivity: we have proved it as the second lemma in the last question. 
\end{proof}\\
(c)
\begin{proof}
    Let $f: X\to Y, g:Y\to X$ be two continuous map. And, $f\circ g \simeq id_Y, g\circ f\simeq id_X$. Let $f^{'} \simeq f$. Then by the lemmas proved in question (a), we get $f^{'} \circ g\simeq id_Y, g\circ f^{'}\simeq id_X$. So, $f^{'}$ is also a homotopy equivalence.
\end{proof}

Chapter 0.4: \\
\begin{proof}
    Let $\iota : A \to X$ be the inclusion map. Let $f= f_1:X\to A , F(x,t)=f_t(x): X\times I \to X$. Then $F$ is a homotopy between $id_X$ and $\iota \circ f$. Let $G= F|_{A\times I}$. Then it is a homotopy between $id_A$ and $f\circ \iota$. So, $\iota$ is a homotopy equivalence.
\end{proof}

Chapter 0.9:\\
\begin{proof}
    Let $X$ be a contractible space. Let A be its retract and $r : A\to X$ be the retract map. Since $X$ is contractible, there exists a map $H: X\times I \to x_0\in X, \ H(X,0)=id_X$. Let $H^{'}= r\big( H|_{A\times I}\big) : A\times I\to A$. And, we have $H^{'}(A,0)=id_A, H^{'}(X,1)=r(x_0)$. This means $A$ can be contracted to $r(x_0)$, i.e. $A$ is contractible.
\end{proof}

Chapter 0.10: \begin{proof}
    (i) '$\longrightarrow$': Since $X$ is contractible, we have $H: X\times I \to X, H(X,0)=id|_X, H(X,1)=x_0\in X$. Let $G=f\circ H: X\times I \to Y$, then $G$ is continuous. $G(X,0)= f\circ H(X,0)= f\circ id_X=f, G(X,1)=f\circ H(X,1)= f(x_0)$. So, $f$ is nullhomotopic.\\
    '$\longleftarrow$': Let $Y=X, f=id|_X$. Then there is a continuous map $H: X\times I \to X, H(X,0)=id_X, H(X,1)=x_0$. So, $X$ is contractible.\\
    (ii) '$\longrightarrow$': Since $X$ is contractible, we have $H: X\times I \to X, H(X,0)=id|_X, H(X,1)=x_0\in X$. We then construct a continuous map $F(y,t)=H(f(y),t)\ : Y\times I \to X$. Then, $F(y,0)=f(y), F(y,1)=x_0$. So, $f$ is nullhomotopic.\\
    '$\longleftarrow$': Like what we have done in (i), we only need to let $Y=X, f=id_X$.
\end{proof}

Chapter 0.11: \begin{proof}
(i) We prove the first half of the problem. $g=id_X\circ d \simeq (h\circ f) \circ g= h \circ (f\circ g) \simeq h \circ id_Y = h$. So, $f\circ g \simeq f \circ h\simeq id_X$. So, $f$ is a homotopy equivalence. \\
(ii) We prove the second half of the problem. If $f\circ g$ and $h\circ f$ are homotopy equivalence, there exists $k_1:Y\to Y,k_2:X\to X$ such that $(f\circ g) \circ k_1 =f\circ (g\circ k_1)\simeq id_Y,\ k_2\circ (h\circ f)=(k_2\circ h) \circ f\simeq id_X$. So, now we can use our first part of conclusion to get $f$ is a homotopy equivalence.
\end{proof}

Chapter 0.20:
\begin{proof}
    Let $X$ be the figure, and it intersects with itself through a circle $C$. First, we can retract the tube conncected to $C$ to a line. This operation creates two point in $X$. First one is $C$, and another one is on the bottom of the klein bottle. Then, we attach these two points together. Now, we can see what lies below the 'point' is a sphere, and there is a circle and a 'neck' connected to the point. In the end, we can retract the 'neck' into a circle once again. So, we get a sphere and two circles who are attached to one point. That is $S^2 \vee S^1 \vee S^1$.
   
    
\end{proof}

Chapter 0.23:
\begin{proof}
    Let this CW complex be $X$, and its two subcomplexes be $A,B$. Let $C=A\cap B $. Then $C$ is also a subcomplex of $X,A,B$. It is because first $C$ is closed, and $X$ is the disjoint union of arbitary open cells $\bigcup\limits_{\alpha \in I}e_\alpha, \{e_\alpha\}=\Gamma$, $A,B$ are also disjoint union of some open cells from $\Gamma$, which makes $C$ must be a disjoint union of some open cells from $\Gamma$. Since $C$ is contractible, $X/C \simeq X, A/C \simeq A, B/C \simeq B$. Since $A/C,B/C$ are still CW complex, they are subcomplexes of $X/C$. Also, they are contractible because $A,B$ are contractible. Then We have $X\simeq X/C \simeq (X/C)/(A/C)$. Let $p : X/C \to (X/C)/(A/C)$ be the quotient map. We claim that $f=p|_{B/C}$ is actually a homeomorphism. First of all, $f$ is continuous. Since $p$ is a quotient map, $p $ is continuous which indicates $p|_{B/C}=f$ is also continuous. And, $p$ is injective. It is because $p$ actually means collapse $A/C$ to one point, which means $p|_{X/C\backslash A/C}$ is injective. Since $B/C \cap A/C$ is acutually an one point set $\{ x_0\}$, which means $f=P|_{B/C}$ is injective. Also, $f=p|_{B/C}$ is surjective by the definition of quotient map. Now we prove $f^{-1}$ is also continuous. It suffices to prove $f$ is an open map. Let $U$ be an open set of $B/C$, then there exists an open subset $U^{'} \subset X/C$ such that $U^{'} \cap B/C =U$. We can rewrite $U^{'}$ as $U^{'}= U^{'}\cap (A/C \backslash B/C) \cup (U^{'}\cap B/C)=U^{'}\cap (A/C \backslash B/C) \cup U$, because $X/C =A/C \cup B/C$. Then $p(U^{'})= p(U) \cup \{p(x_0)\}$. This is a disjoint union iff $x_0 \notin U, U^{'}\cap (A/C \backslash B/C)\neq \emptyset $. If $x_0 \notin U$, then $U= X/C \backslash A/C \cap U^{{'}}$. So, $U$ itself is an open subset of $X/C$. So, $f(U)=p|_{B/C}(U)=p(U)$ is an open subset of $(X/C)/A/C$, and since $p(U) \subset p(B/C)$ it is also an open subset of $p(B/C)$. Then we assume $x_0 \in U$. So, $f(U)=p(U)=p(U^{'})$ by the discussion before. So, $p(U)$ is an open subset of $X/C=p(B/C)$ as $p(U^{'})$ is an open subset. Now, we can conclude that $(X/C)/(A/C)$ is contractible, since it is homeomorphic to $B/C$ which is contractible($B/C\simeq B$). So, $X\simeq X/C \simeq (X/C)/(A/C)$ is also contracible. Last, we only need to verify homeomorphism is indeed a homotopy equivalence. It is trivial since a homeomorphism $f$ always have $f\circ f^{-1}=id, f^{-1} \circ f =id$.
\end{proof}

Chapter 0.28:
\begin{proof}
    Suppose we have a continuous map $g: X_1 \sqcup _fX_0 \to Y$, and a homotopy $H: X_0 \times I \to Y$ which satisfies $H(x_0,0)= g|_{X_0}$.
    Let $p:X_1 \sqcup X_0 \to X_1 \sqcup_f X_0$ be the quotient map. And let $g^{'}= g \circ p$, and a homotopy $H^{'}= H(f(a),t)\ : A\times I \to Y$. Then $H^{'}(a,0)=H(f(a),0)=g(f(a))=g^{'}|_{A}, a\in A$. By the homotopy extension property, we can get another homotopy $H^{''}: X_1 \times I \to Y$ which satisfies $H^{''}(a,t)=H^{'}(a,t).$ Now we need to glue these two homotopies together. We define $F(p(x),t)=\begin{cases}
        H^{''}(x,t), x\in X_1\\
        H(x,t),\ x\in X_0
    \end{cases}$. This is well defined, and also continuous. Because we can check its continuity in $p(X_1)\times I, p(X_0)\times I$, who both are closed in $(X_1 \sqcup_fX_0 )\times I$.
\end{proof}
\end{document}
